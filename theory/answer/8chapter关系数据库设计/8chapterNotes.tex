
\documentclass{dingjia}

\ctexset{autoindent=false}
\setlength\parindent{0pt}

\begin{document}


\chapter{关系数据库设计}

\section{好的关系设计的特点}

函数依赖 functional dependency, $dept_{name} \rightarrow budget$

有损分解 lossy decomposition, 无损分解 lossless decomposition

\section{原子域和第一范式}

一个域是原子的 atomic,如果该域的元素被认为是不可分的单元。如果关系模式 R 的所
有属性的域都是原子的,那 R 属于第一范式。

使用以集合为值的属性会导致冗余存储数据的设计,进而会导致不一致。如为一个教师
存储一个课程标识号的集合,并为每一课程存储一个教师标识号的集合。数据变动时,
必须两处都变动,不然数据库可能处于不一致的状态。

另外,如果只保留其中一个集合,将避免重复的值,然而,只保留其中一个集合将使查
询变得复杂,并且也不好决定究竟保留哪一个。

在许多含有复杂结构的实体域中,强制使用第一范式反而会给应用程序带来不必要的负
担。

\section{使用函数依赖进行分解}

\subsection{码和函数依赖}

本书表示法:
\begin{enumerate}
\item 一般情况下,用小写希腊字母表示属性集(也可能是模式,也有可能不是
  );r(R)表示该关系模式是关系 r 的,R 表示属性集。不过我们常只用 R 来简化表示。
  一个关系模式必然是一个属性集,但并非所有的属性集都是模式。
\item K 是 r(R)的超码。

\item  实例:一个关系在任意给定时间内都有特定的值。如关系 r 的实例。
\end{enumerate}

一个关系满足所有现实约束的实例,被称为关系的合法实例 legal instance.

重新定义超码K概念如下:给定关系 r(R),其任意合法实例中的元组对 $t_1$ 和 $t_2$ 总
满足,若 $t_1 \neq t_2$ ,则 $t_1[K] \neq t_2[K]$ 。显然,如果 r 中没有两条元组
在 K 上具有相同的值,那么在 r 中一个 K(超码)值唯一标识一条元组。

考虑关系模式r(R),另 $\alpha \subseteq R \text{ and } \beta \subseteq R$:
\begin{enumerate}
\item 给定r(R)的一个实例,如果实例中任意元组对 $t_1,t_2$,若 $t_1[\alpha] = t_2[\alpha]$,
  则$t_1[\beta] = t_2[\beta]$,那么这个实例满足satisfy函数依赖 $\alpha \rightarrow \beta$。

\item 如果r(R)所有实例都满足函数依赖 $\alpha \rightarrow \beta$,则我们说函
  数依赖在r(R)上成立。
\end{enumerate}

使用函数依赖这一概念,我们说如果函数依赖 $K \rightarrow R$ 在r(R)上成立,则K
是r(R)的一个超码。

函数依赖使我们可以表示不能用超码表示的约束。
\[ inst\_dept ( ID, name, salary, dept\_name, building, budget)\]
\[ ID, dept\_name \rightarrow name, salary, building, budget \]

有些函数依赖被称为平凡的trivial,因为它们在所有关系中都满足。一般地,如果 $
\beta \subseteq \alpha$,则形如 $\alpha \rightarrow \beta$的函数依赖是平凡的。
重要的是,要认识到一个关系实例可能满足某些函数依赖,他们并不需要在关系的模式
上成立。如函数依赖可以只对应于关系的某个子属性集。

F集合的闭包closure $F^+$,也就是能够从给定F集合推导出的所有函数依赖的集合。

\subsection{Boyce-Codd范式}

我们能够达到的比较满意的范式之一是Boyce-Codd Normal Form,BCNF。它消除所有基
于函数依赖能够发现的冗余。

具有函数依赖集F的关系模式R属于BCNF的条件是,对 $F^+$ 中所有形如 $\alpha
\rightarrow \beta$ 的函数依赖,下面至少有一项成立:
\begin{itemize}
\item $\alpha \rightarrow \beta$ 是平凡的函数依赖(即 $\beta \subseteq \alpha$
\item $\alpha$ 是模式R的一个超码
\end{itemize}

一个数据库设计属于BCNF的条件是,构成该设计的关系模式集中的每个模式都属于BCNF。

设R为不属于BCNF的一个模式,则存在至少一个非平凡的函数依赖 $\alpha
\rightarrow \beta$,其中 $\alpha$ 不是R的超码。我们可在设计中用两个模式取代R:
\begin{itemize}
\item $ (\alpha \cup \beta)$
\item $ (R - (\beta - \alpha))$
\end{itemize}

在 $inst\_dept ( ID, name, salary, dept\_name, building, budget)$ 中,
$\alpha = dept\_name \quad \beta = {building, budget}$ , inst\_dept被取代为
\begin{itemize}
\item $(\alpha \cup \beta) = (dept\_name, building, budget)$
\item $(R - (\beta - \alpha)) = (ID, name, salary, dept\_name)$
\end{itemize}
在这个例子中,$\beta - \alpha = \beta$。

\subsection{BCNF和保持依赖}

我们已经看到多种表达数据库一致性约束的方式:主码约束、函数依赖、CHECK约束,
断言和触发器。每次数据库更新时检查这些约束的开销很大,因此需要将其设计为能够
高效检查的约束。

其实按照定义任何只包含两个属性的模式都属于BCNF。

并与我们的设计使得该函数依赖的强制实施在计算上很困难,因此我们的设计不是保持
依赖的,dependency preserving.

\subsection{第三范式}

BCNF要求所有非平凡函数依赖都形如 $\alpha \rightarrow \beta$,其中\alpha为一
个超码。3CNF放宽了这个约束,它允许左侧不是超码的某些非平凡的函数依赖.
\begin{description}
\item[第三范式] 对于 $F^+$中所有形如 $\alpha \rightarrow \beta$ 的函数依赖,以下至少有一
  项成立:
  \begin{itemize}
  \item $\alpha \rightarrow \beta$ 是一个平凡的函数依赖
  \item $\alpha$ 是R中的一个超码
  \item $\beta - \alpha$ 中的每个属性A都包含于R的\textbf{某一个}候选码中
  \end{itemize}
\end{description}

只满足3NF定义中第三个条件的依赖 $\alpha \rightarrow \beta$ 在BCNF中是不成立
的。

\subsection{更高的范式}

\section{函数依赖理论}

\subsection{函数依赖集的闭包}

给定模式上的函数依赖集F,我们可以证明某些其他的函数依赖在模式上也成立。我们
称这些函数依赖被F“逻辑蕴含”。当检验范式时,只考虑给定的函数依赖集是不够的;
除此之外,还需要考虑模式上成立的所有函数依赖。

更正式地,给定关系模式r(R),如果r(R)的每一个满足F的实例也满足f,则R上的函数
依赖f被r上的函数依赖集F逻辑蕴含logically imply.

另F为一个函数依赖集。F的闭包是被F逻辑蕴含的所有函数依赖的集合,记作 $F^+$.给
定F,可以由函数依赖的形式化定义直接计算出 $F^+$.

Armstrong公理Armstrong's axiom:
\begin{itemize}
\item[自反律 reflexivity rule] 若 $\alpha$ 为一个属性集,且 $\beta \subseteq
  \alpha$,则$\alpha \rightarrow \beta$ 成立。
\item[增补律 augmentation rule] 若 $\alpha \rightarrow \beta$ 成立且$\gamma$
  为一属性集,则 $\gamma\alpha \rightarrow \gamma\beta$成立
\item[传递律 transitivity rule] 若 $\alpha \rightarrow \beta$ 和 $\beta
  \rightarrow \gamma$ 成立,则 $\alpha \rightarrow \gamma$ 成立。
\end{itemize}

虽然 Armstrong公理是完备的,但直接用它们计算$F^+$会很麻烦,可使用推导出的其他规则
\begin{itemize}
\item[合并律 union rule] 若 $\alpha \rightarrow \beta$ 和 $\alpha
  \rightarrow \gamma$ 成立,则 $\alpha \rightarrow \beta\gamma$ 成立。
\item[分解律 decomposition] 若 $\alpha \rightarrow \beta\gamma$成立,
  则 $\alpha \rightarrow \beta$ 和 $\alpha \rightarrow \gamma$ 成立。
\item[伪传递律 pseudotransitivity rule] 若 $\alpha$ 为一个属性集,且 $\beta
  \subseteq \alpha$,则$\alpha \rightarrow \beta$ 成立。
\end{itemize}

\subsection{属性集的闭包}

$\alpha \rightarrow \beta$, $\beta$被$\alpha$函数确定functionally determine


\end{document}